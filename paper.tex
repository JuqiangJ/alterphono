acoustic and articulary models

We propose a new method for phoneme ("segment") distance.

Distance is historically calculated with X methods:

- ad-hoc values, established by researchers based on personal experience
- models that quantify phoneme descriptions is some way, for example by calculating an
euclidian distance for vowels in the vowel trapezium
- measures derived from acoustic properties, usually combined with some frequrency
information for balancing (such as in Mielke)
- measures based in the distribution of phonemes, and eventually allophones, in a single
language or across language inventories

when considering phoneme properties, there are X ways of considering them:
- not considering them at all, i.e., considering each phoneme as purely independent and
as equally distant from every other single phoneme
- considering the similarities using the more common descriptions, as used in IPA,
mostly based in place and manner of articulation; this can either be considered as dummy
variables (such that the distance between different properties is equal, like from stop to
fricative or from stop to affricate) or a metric can be established by the researcher
(easier for place of articulation, as it can more easily be described as a continous
variable)
- considering distinctive features, which are usually binary but can likely also be set
to zero (which can be considered as a zero, a distinct value, or defaulting to minus);
it is however important to remember that this can affect the consideration of sequences,
such as diphtong, because sometimes some values are duplicated or triplicated

we decided to base our metric in the global distribution of segments across language
inventories and allophones for a given phoneme; in particular, we assume that:

- accept that phonemic systems tend do balance between maximal perceptual distinctiveness
and minimal speaker effort /JanWillemVanLeussenMA2009.pdf
- the more two phonemes are co-occur across inventories, the more different they are
- the more two phonemes co-occur as allophones across languages, the more similar
they are
- the distance between a phoneme and itself should always be zero
- a single system should consider all segments found in languages, including vowels,
consonants, diphtongs, etc.
- similar phonemes co-occurr: this is given by the tendency for each voiceless consonant to
have its voiced counterpart, for example, or for a vowel to have rounded/unrounded; it
also considers that it is unlikely for a language to have a single phoneme for a given place
or manner of articulation (for example, only one bilabial consonant, or only one high vowel)
